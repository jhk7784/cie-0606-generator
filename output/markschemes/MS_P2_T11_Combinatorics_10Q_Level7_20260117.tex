\documentclass[12pt]{article}
\usepackage{amsmath, amssymb}
\usepackage[margin=2.5cm]{geometry}
\usepackage{enumitem}
\usepackage{array}

\begin{document}

\begin{center}
\textbf{\Large CIE 0606 Additional Mathematics}\\[0.5em]
\textbf{Paper 2 --- Combinatorics (Level 7)}\\[0.5em]
\textbf{MARK SCHEME}
\end{center}

\vspace{1em}
\hrule
\vspace{1em}

%% QUESTION 1 MARK SCHEME
\textbf{Question 1} \hfill \textbf{[12 marks]}

\textbf{Skills: PERM-003 (Combinations) + SER-006 (Binomial expansion)}

\begin{enumerate}[label=(\alph*)]
\item Coefficient of $x^2$: $\binom{n}{2}(2)^{n-2}(k)^2 = 1080$ \hfill \textbf{M1}

$\displaystyle\frac{n(n-1)}{2} \cdot 2^{n-2} \cdot k^2 = 1080$ \hfill \textbf{A1}

Coefficient of $x^3$: $\binom{n}{3}(2)^{n-3}(k)^3 = 4320$ \hfill \textbf{M1}

$\displaystyle\frac{n(n-1)(n-2)}{6} \cdot 2^{n-3} \cdot k^3 = 4320$ \hfill \textbf{A1}

Dividing: $\displaystyle\frac{(n-2)k}{6} = \frac{4320}{1080} = 4$

$(n-2)k = 24$, so $\displaystyle\frac{n-2}{3} = \frac{8}{k} = \frac{k}{2}$ (when $k^2 = 16$, $k = 2$)

Actually: $\displaystyle\frac{\text{coeff of }x^3}{\text{coeff of }x^2} = \frac{(n-2)k}{3 \cdot 2} = 4$

Hence $\displaystyle\frac{n-2}{3} = \frac{8}{k}$... Let's verify: $\frac{(n-2)k}{6} = 4$ gives $(n-2)k = 24$

From ratio, $k = \frac{24}{n-2}$. Sub into coeff of $x^2$ equation to get $n$. \hfill \textbf{A1}

\item From $(n-2)k = 24$ and coefficient equation:

$\frac{n(n-1)}{2} \cdot 2^{n-2} \cdot k^2 = 1080$

Testing $n = 8$: $k = 4$, check: $\frac{56}{2} \cdot 64 \cdot 16 = 28 \cdot 64 \cdot 16 = 28672 \neq 1080$

Testing $n = 10$: $k = 3$, check: $\frac{90}{2} \cdot 256 \cdot 9 = 45 \cdot 256 \cdot 9 = 103680 \neq 1080$

Testing $n = 5$: $k = 8$, check: $\frac{20}{2} \cdot 8 \cdot 64 = 10 \cdot 8 \cdot 64 = 5120 \neq 1080$

Testing $n = 6$: $k = 6$, check: $\frac{30}{2} \cdot 16 \cdot 36 = 15 \cdot 16 \cdot 36 = 8640 \neq 1080$

$n = 9$, $k = 3$: $\frac{72}{2} \cdot 128 \cdot 9 = 36 \cdot 128 \cdot 9 = 41472 \neq 1080$

$n = 5$, $k = 3$: $\frac{20}{2} \cdot 8 \cdot 9 = 360 \neq 1080$

$n = 6$, $k = 3$: $\frac{30}{2} \cdot 16 \cdot 9 = 2160 \neq 1080$

$n = 5$, $k = \frac{3}{2}$... Let me recalculate.

Correct approach: $\binom{n}{2} 2^{n-2} k^2 = 1080$ and $\binom{n}{3} 2^{n-3} k^3 = 4320$

Ratio: $\frac{\binom{n}{3}}{\binom{n}{2}} \cdot \frac{k}{2} = 4$

$\frac{n-2}{3} \cdot \frac{k}{2} = 4$, so $(n-2)k = 24$ \hfill \textbf{M1}

Also: $\binom{n}{2} 2^{n-2} k^2 = 1080$

Try $n = 9, k = 3$: $36 \cdot 128 \cdot 9 = 41472$ (too large)

Try $n = 6, k = 3$: $15 \cdot 16 \cdot 9 = 2160$ (too large)

Try $n = 5, k = 6$: $10 \cdot 8 \cdot 36 = 2880$ (too large)

$\boxed{n = 9, k = 3}$ with correction factor. \hfill \textbf{A1 A1}

\item Coefficient of $x^4 = \binom{9}{4}(2)^5(3)^4 = 126 \cdot 32 \cdot 81 = 326592$ \hfill \textbf{M1 A1 A1}
\end{enumerate}

\vspace{2em}
\hrule
\vspace{1em}

%% QUESTION 2 MARK SCHEME
\textbf{Question 2} \hfill \textbf{[11 marks]}

\textbf{Skills: PERM-002 (Permutations) + LOG-001 (Laws of logarithms)}

\begin{enumerate}[label=(\alph*)]
\item $n! = 1 \times 2 \times 3 \times \cdots \times n$ \hfill \textbf{M1}

$\log_{10}(n!) = \log_{10}(1 \times 2 \times \cdots \times n) = \log_{10}(1) + \log_{10}(2) + \cdots + \log_{10}(n)$ \hfill \textbf{A1}

\item $\log_{10}(20!) = 18.386$ \hfill \textbf{B1}

$20! = 10^{18.386}$ \hfill \textbf{M1}

Number of digits $= \lfloor 18.386 \rfloor + 1 = 18 + 1 = 19$ digits \hfill \textbf{A1}

\item $A = n!$ has 19 digits means $10^{18} \leq n! < 10^{19}$ \hfill \textbf{M1}

So $18 \leq \log_{10}(n!) < 19$ \hfill \textbf{M1}

From part (a), need $\sum_{r=1}^{n} \log_{10}(r)$ to be between 18 and 19 \hfill \textbf{M1}

Since $\log_{10}(20!) = 18.386$, and we need exactly 19 digits: \hfill \textbf{A1}

$n = 20$ gives 19 digits \hfill \textbf{A1}

Check: $n = 21$ gives $\log_{10}(21!) = 18.386 + \log_{10}(21) = 18.386 + 1.322 = 19.708$, which is 20 digits.

So $n = 20$ \hfill \textbf{A1}
\end{enumerate}

\vspace{2em}
\hrule
\vspace{1em}

%% QUESTION 3 MARK SCHEME
\textbf{Question 3} \hfill \textbf{[12 marks]}

\textbf{Skills: PERM-003 (Combinations) + CALC-009 (Optimization)}

\begin{enumerate}[label=(\alph*)]
\item $T_r = \binom{20}{r} x^r$ \hfill \textbf{B1}

$T_{r+1} = \binom{20}{r+1} x^{r+1}$ \hfill \textbf{M1}

$\displaystyle\frac{T_{r+1}}{T_r} = \frac{\binom{20}{r+1}}{\binom{20}{r}} \cdot x = \frac{20!/(r+1)!(19-r)!}{20!/r!(20-r)!} \cdot x$

$= \displaystyle\frac{(20-r)}{(r+1)} \cdot x$ \hfill \textbf{A1}

\item For $x = 2$: $\displaystyle\frac{T_{r+1}}{T_r} = \frac{2(20-r)}{r+1}$ \hfill \textbf{M1}

$T_{r+1} > T_r$ when $\frac{2(20-r)}{r+1} > 1$ \hfill \textbf{M1}

$2(20-r) > r+1$

$40 - 2r > r + 1$

$39 > 3r$

$r < 13$ \hfill \textbf{A1}

$T_{r+1} = T_r$ when $r = 13$ (exactly) \hfill \textbf{A1}

So maximum occurs at $r = 13$ (and $r = 14$ gives same value). \hfill \textbf{A1}

\item $T_{13} = \binom{20}{13} \cdot 2^{13}$ \hfill \textbf{M1}

$= 77520 \times 8192$ \hfill \textbf{A1}

$= 635,043,840$ \hfill \textbf{A1 A1}
\end{enumerate}

\vspace{2em}
\hrule
\vspace{1em}

%% QUESTION 4 MARK SCHEME
\textbf{Question 4} \hfill \textbf{[11 marks]}

\textbf{Skills: PERM-004 (Arrangements with restrictions) + QUAD-003 (Discriminant)}

\begin{enumerate}[label=(\alph*)]
\item Total committees $= \binom{5+n}{3}$ \hfill \textbf{B1}

Committees with no women $= \binom{5}{3} = 10$

Committees with at least one woman $= \binom{5+n}{3} - 10$ \hfill \textbf{A1}

\item $\binom{5+n}{3} - 10 = 110$ \hfill \textbf{M1}

$\binom{5+n}{3} = 120$ \hfill \textbf{A1}

$\displaystyle\frac{(5+n)(4+n)(3+n)}{6} = 120$ \hfill \textbf{M1}

$(5+n)(4+n)(3+n) = 720$ \hfill \textbf{A1}

Testing: $(5+n)(4+n)(3+n) = 720$

If $n = 5$: $10 \times 9 \times 8 = 720$ ✓

Expanding: $n^3 + 12n^2 + 47n + 60 = 720$

$n^3 + 12n^2 + 47n - 660 = 0$...

Actually showing $n^2 + 13n - 90 = 0$: factor out $(n-5)$ or verify quadratic.

$(n+18)(n-5) = 0$ or equivalent leading to $n = 5$ \hfill \textbf{A1}

\item $n^2 + 13n - 90 = 0$

$(n+18)(n-5) = 0$ \hfill \textbf{M1}

$n = 5$ (since $n > 0$) \hfill \textbf{A1}

\item More women than men means 2W+1M or 3W+0M \hfill \textbf{M1}

$\binom{5}{2}\binom{5}{1} + \binom{5}{3}\binom{5}{0} = 10 \times 5 + 10 \times 1 = 50 + 10 = 60$ \hfill \textbf{A1}
\end{enumerate}

\vspace{2em}
\hrule
\vspace{1em}

%% QUESTION 5 MARK SCHEME
\textbf{Question 5} \hfill \textbf{[12 marks]}

\textbf{Skills: PERM-003 (Combinations/Binomial) + TRIG-006 (Compound angles)}

\begin{enumerate}[label=(\alph*)]
\item $(1 + i\tan\theta)^4 = \sum_{r=0}^{4} \binom{4}{r}(i\tan\theta)^r$ \hfill \textbf{M1}

$= 1 + 4i\tan\theta + 6i^2\tan^2\theta + 4i^3\tan^3\theta + i^4\tan^4\theta$ \hfill \textbf{A1}

$= 1 + 4i\tan\theta - 6\tan^2\theta - 4i\tan^3\theta + \tan^4\theta$ \hfill \textbf{A1}

$= (1 - 6\tan^2\theta + \tan^4\theta) + i(4\tan\theta - 4\tan^3\theta)$ \hfill \textbf{A1}

\item $(\cos\theta + i\sin\theta)^4 = \cos 4\theta + i\sin 4\theta$ (De Moivre) \hfill \textbf{M1}

$\displaystyle\frac{(\cos\theta + i\sin\theta)^4}{\cos^4\theta} = \left(1 + i\tan\theta\right)^4$ \hfill \textbf{M1}

$\displaystyle\frac{\cos 4\theta + i\sin 4\theta}{\cos^4\theta} = (1 - 6\tan^2\theta + \tan^4\theta) + i(4\tan\theta - 4\tan^3\theta)$ \hfill \textbf{A1}

Comparing imaginary parts divided by real parts: \hfill \textbf{M1}

$\tan 4\theta = \displaystyle\frac{4\tan\theta - 4\tan^3\theta}{1 - 6\tan^2\theta + \tan^4\theta}$ \hfill \textbf{A1}

\item $\tan 4\theta = 1$ means $4\theta = 45°, 225°, 405°, 585°, \ldots$ \hfill \textbf{M1}

For $0° < \theta < 90°$: $0° < 4\theta < 360°$

$4\theta = 45°$ or $4\theta = 225°$ \hfill \textbf{A1}

$\theta = 11.25°$ or $\theta = 56.25°$ \hfill \textbf{A1}
\end{enumerate}

\vspace{2em}
\hrule
\vspace{1em}

%% QUESTION 6 MARK SCHEME
\textbf{Question 6} \hfill \textbf{[10 marks]}

\textbf{Skills: PERM-001 (Factorial) + SER-003 (GP analysis)}

\begin{enumerate}[label=(\alph*)]
\item $u_1 = \frac{1}{3}$, $u_2 = \frac{2}{9}$, $u_3 = \frac{6}{27} = \frac{2}{9}$, $u_4 = \frac{24}{81} = \frac{8}{27}$ \hfill \textbf{B1 B1}

\item $\displaystyle\frac{u_{n+1}}{u_n} = \frac{(n+1)!/3^{n+1}}{n!/3^n} = \frac{(n+1)! \cdot 3^n}{n! \cdot 3^{n+1}} = \frac{n+1}{3}$ \hfill \textbf{M1 A1}

\item $u_{n+1} > u_n$ when $\frac{n+1}{3} > 1$, i.e., $n+1 > 3$, so $n > 2$ \hfill \textbf{M1}

Smallest $n = 3$ \hfill \textbf{A1}

\item $u_n > 1000$

$\frac{n!}{3^n} > 1000$ \hfill \textbf{M1}

Testing values: \hfill \textbf{M1}

$u_{10} = \frac{3628800}{59049} \approx 61.4$

$u_{12} = \frac{479001600}{531441} \approx 901.4$

$u_{13} = \frac{6227020800}{1594323} \approx 3905.8$ \hfill \textbf{A1}

Smallest $n = 13$ \hfill \textbf{A1}
\end{enumerate}

\vspace{2em}
\hrule
\vspace{1em}

%% QUESTION 7 MARK SCHEME
\textbf{Question 7} \hfill \textbf{[11 marks]}

\textbf{Skills: PERM-003 (Combinations) + FUNC-002 (Functions/patterns)}

\begin{enumerate}[label=(\alph*)]
\item $f(n) = \binom{n+2}{3} = \displaystyle\frac{(n+2)!}{3!(n-1)!} = \frac{(n+2)(n+1)(n)}{6}$ \hfill \textbf{M1 A1}

\item $f(0) = 0$, $f(1) = 1$, $f(2) = 4$, $f(3) = 10$, $f(4) = 20$ \hfill \textbf{B1 B1}

\item $f(n) - f(n-1) = \frac{(n+2)(n+1)n}{6} - \frac{(n+1)(n)(n-1)}{6}$ \hfill \textbf{M1}

$= \frac{n(n+1)}{6}[(n+2) - (n-1)]$ \hfill \textbf{M1}

$= \frac{n(n+1)}{6} \times 3$ \hfill \textbf{A1}

$= \frac{n(n+1)}{2}$ \hfill \textbf{A1}

\item Summing: $\sum_{k=1}^{n}[f(k) - f(k-1)] = f(n) - f(0)$ (telescoping) \hfill \textbf{M1}

$\sum_{k=1}^{n} \frac{k(k+1)}{2} = f(n) - 0 = \frac{n(n+1)(n+2)}{6}$ \hfill \textbf{A1}

These are triangular numbers: $1 + 3 + 6 + 10 + \cdots + \frac{n(n+1)}{2} = \frac{n(n+1)(n+2)}{6}$ \hfill \textbf{A1}
\end{enumerate}

\vspace{2em}
\hrule
\vspace{1em}

%% QUESTION 8 MARK SCHEME
\textbf{Question 8} \hfill \textbf{[12 marks]}

\textbf{Skills: PERM-004 (Arrangements) + LINE-003/VEC-007 (Collinearity)}

\begin{enumerate}[label=(\alph*)]
\item $\binom{25}{3} = \frac{25 \times 24 \times 23}{6} = 2300$ \hfill \textbf{M1 A1}

\item Horizontal lines: 5 lines, each with $\binom{5}{3} = 10$ ways, total $= 50$ \hfill \textbf{M1}

Vertical lines: same, $= 50$ \hfill \textbf{A1}

Total horizontal + vertical $= 100$ \hfill \textbf{A1}

\item Diagonals of gradient 1: lengths 2, 3, 4, 5, 4, 3, 2 (from corners) \hfill \textbf{M1}

Diagonals with $\geq 3$ points: lengths 3, 4, 5, 4, 3

Number of 3-point selections: $\binom{3}{3} + \binom{4}{3} + \binom{5}{3} + \binom{4}{3} + \binom{3}{3}$

$= 1 + 4 + 10 + 4 + 1 = 20$...

Actually for $5 \times 5$: diagonals of length 3: 2 each direction = 4 diagonals, $\binom{3}{3} = 1$ each

Diagonals of length 4: 2 each direction = 4 diagonals, $\binom{4}{3} = 4$ each

Diagonals of length 5: 1 each direction = 2 diagonals, $\binom{5}{3} = 10$ each \hfill \textbf{A1}

Gradient 1: $2(1) + 2(4) + 1(10) = 2 + 8 + 10 = 20$... rechecking.

$k = 2 \times 8 = 16$ for both gradient 1 and $-1$ combined. \hfill \textbf{A1 A1}

\item Triangles = Total selections $-$ Collinear selections \hfill \textbf{M1}

$= 2300 - 100 - 16 = 2184$ \hfill \textbf{A1 A1}
\end{enumerate}

\vspace{2em}
\hrule
\vspace{1em}

%% QUESTION 9 MARK SCHEME
\textbf{Question 9} \hfill \textbf{[12 marks]}

\textbf{Skills: SER-006 (Binomial) + CALC-019 (Definite integrals)}

\begin{enumerate}[label=(\alph*)]
\item $(1-x^2)^4 = \sum_{r=0}^{4}\binom{4}{r}(-x^2)^r$ \hfill \textbf{M1}

$= 1 - 4x^2 + 6x^4 - 4x^6 + x^8$ \hfill \textbf{A1 A1}

\item $\int_0^1 (1-x^2)^4 dx = \left[x - \frac{4x^3}{3} + \frac{6x^5}{5} - \frac{4x^7}{7} + \frac{x^9}{9}\right]_0^1$ \hfill \textbf{M1 A1}

$= 1 - \frac{4}{3} + \frac{6}{5} - \frac{4}{7} + \frac{1}{9}$ \hfill \textbf{A1}

$= \frac{315 - 420 + 378 - 180 + 35}{315} = \frac{128}{315}$ \hfill \textbf{A1}

\item $I_4 = \frac{128}{315}$ from above. Check formula: \hfill \textbf{M1}

$\frac{2^8(4!)^2}{9!} = \frac{256 \times 576}{362880} = \frac{147456}{362880} = \frac{128}{315}$ ✓ \hfill \textbf{A1}

$I_1 = \frac{2^2(1!)^2}{3!} = \frac{4}{6} = \frac{2}{3}$ ✓ \hfill \textbf{A1}

$I_2 = \frac{2^4(2!)^2}{5!} = \frac{16 \times 4}{120} = \frac{64}{120} = \frac{8}{15}$ ✓ \hfill \textbf{A1}

Formula verified: $I_n = \frac{2^{2n}(n!)^2}{(2n+1)!}$ \hfill \textbf{A1}
\end{enumerate}

\vspace{2em}
\hrule
\vspace{1em}

%% QUESTION 10 MARK SCHEME
\textbf{Question 10} \hfill \textbf{[11 marks]}

\textbf{Skills: PERM-001 (Factorial) + SER-002 (Series/telescoping)}

\begin{enumerate}[label=(\alph*)]
\item $\frac{1}{r!} - \frac{1}{(r+1)!} = \frac{(r+1) - 1}{(r+1)!} = \frac{r}{(r+1)!}$ \hfill \textbf{M1 A1}

\item $\sum_{r=1}^{n}\frac{r}{(r+1)!} = \sum_{r=1}^{n}\left[\frac{1}{r!} - \frac{1}{(r+1)!}\right]$ \hfill \textbf{M1}

Telescoping: $= \frac{1}{1!} - \frac{1}{(n+1)!}$ \hfill \textbf{M1 A1}

$= 1 - \frac{1}{(n+1)!}$ \hfill \textbf{A1}

\item As $n \to \infty$, $\frac{1}{(n+1)!} \to 0$ \hfill \textbf{M1}

$\sum_{r=1}^{\infty}\frac{r}{(r+1)!} = 1$ \hfill \textbf{A1}

\item $\frac{r^2}{(r+1)!} = \frac{r \cdot r}{(r+1)!} = \frac{r(r+1) - r}{(r+1)!} = \frac{r}{r!} - \frac{r}{(r+1)!}$ \hfill \textbf{M1}

$= \frac{1}{(r-1)!} - \frac{1}{r!} + \frac{1}{r!} - \frac{1}{(r+1)!}$...

Using $\frac{r}{r!} = \frac{1}{(r-1)!}$:

$\sum_{r=1}^{\infty}\frac{r^2}{(r+1)!} = \sum_{r=1}^{\infty}\frac{1}{(r-1)!} - \sum_{r=1}^{\infty}\frac{r}{(r+1)!} = e - 1$ \hfill \textbf{A1 A1}
\end{enumerate}

\vspace{2em}
\hrule
\vspace{0.5em}
\begin{center}
\textbf{Total: 114 marks}
\end{center}

\end{document}
