\documentclass[12pt]{article}
\usepackage{amsmath, amssymb}
\usepackage[margin=2.5cm]{geometry}
\usepackage{enumitem}

\begin{document}

\begin{center}
\textbf{\Large CIE 0606 Additional Mathematics}\\[0.5em]
\textbf{Paper 2 --- Combinatorics (Level 7)}\\[0.5em]
\textit{10 Questions --- Cross-Topic Integration}
\end{center}

\vspace{1em}
\hrule
\vspace{1em}

%% QUESTION 1: Combinations + Binomial Expansion
\textbf{Question 1} \hfill \textbf{[12 marks]}

In the expansion of $(2 + kx)^n$, where $n$ is a positive integer and $k$ is a constant, the coefficient of $x^2$ is 1080 and the coefficient of $x^3$ is 4320.

\begin{enumerate}[label=(\alph*)]
\item Show that $\displaystyle \frac{n-2}{3} = \frac{k}{2}$. \hfill [5]
\item Find the values of $n$ and $k$. \hfill [4]
\item Hence find the coefficient of $x^4$ in the expansion. \hfill [3]
\end{enumerate}

\vspace{2em}

%% QUESTION 2: Permutations + Logarithms
\textbf{Question 2} \hfill \textbf{[11 marks]}

\begin{enumerate}[label=(\alph*)]
\item Show that $\log_{10}(n!) = \log_{10}(1) + \log_{10}(2) + \log_{10}(3) + \cdots + \log_{10}(n)$. \hfill [2]
\item Given that $\log_{10}(20!) = 18.386$ (to 3 decimal places), find the number of digits in $20!$. \hfill [3]
\item The number of ways of arranging $n$ distinct objects in a row is $A$. Given that $A$ has exactly 19 digits, find the value of $n$. \hfill [6]
\end{enumerate}

\vspace{2em}

%% QUESTION 3: Combinations + Calculus (Optimization)
\textbf{Question 3} \hfill \textbf{[12 marks]}

The general term in the expansion of $(1 + x)^{20}$ is $T_r = \binom{20}{r}x^r$ for $r = 0, 1, 2, \ldots, 20$.

\begin{enumerate}[label=(\alph*)]
\item Show that $\displaystyle \frac{T_{r+1}}{T_r} = \frac{(20-r)x}{r+1}$. \hfill [3]
\item For $x = 2$, find the value of $r$ for which $T_r$ is greatest. \hfill [5]
\item Find this greatest term. \hfill [4]
\end{enumerate}

\vspace{2em}

%% QUESTION 4: Arrangements + Quadratic Discriminant
\textbf{Question 4} \hfill \textbf{[11 marks]}

A committee of 3 people is to be chosen from 5 men and $n$ women, where $n \geq 2$.

\begin{enumerate}[label=(\alph*)]
\item Show that the number of ways of choosing the committee if it must contain at least one woman is $\binom{5+n}{3} - 10$. \hfill [2]
\item Given that the number of committees with at least one woman is 110, show that $n^2 + 13n - 90 = 0$. \hfill [5]
\item Hence find the value of $n$. \hfill [2]
\item Find the number of committees that contain more women than men. \hfill [2]
\end{enumerate}

\vspace{2em}

%% QUESTION 5: Combinations + Trigonometry
\textbf{Question 5} \hfill \textbf{[12 marks]}

\begin{enumerate}[label=(\alph*)]
\item Use the binomial theorem to expand $(1 + i\tan\theta)^4$, where $i = \sqrt{-1}$, giving your answer in the form $a + bi$ where $a$ and $b$ are expressions in $\tan\theta$. \hfill [4]
\item By considering $(\cos\theta + i\sin\theta)^4$ divided by $\cos^4\theta$, show that \\
$\tan 4\theta = \displaystyle\frac{4\tan\theta - 4\tan^3\theta}{1 - 6\tan^2\theta + \tan^4\theta}$. \hfill [5]
\item Hence solve $\tan 4\theta = 1$ for $0° < \theta < 90°$. \hfill [3]
\end{enumerate}

\vspace{2em}

%% QUESTION 6: Factorial Patterns + Geometric Progression
\textbf{Question 6} \hfill \textbf{[10 marks]}

Let $u_n = \displaystyle\frac{n!}{3^n}$ for positive integers $n$.

\begin{enumerate}[label=(\alph*)]
\item Find the exact values of $u_1$, $u_2$, $u_3$, and $u_4$. \hfill [2]
\item Show that $\displaystyle\frac{u_{n+1}}{u_n} = \frac{n+1}{3}$. \hfill [2]
\item Find the smallest value of $n$ for which $u_{n+1} > u_n$. \hfill [2]
\item Find the smallest value of $n$ for which $u_n > 1000$. \hfill [4]
\end{enumerate}

\vspace{2em}

%% QUESTION 7: Combinations + Functions
\textbf{Question 7} \hfill \textbf{[11 marks]}

A function $f$ is defined such that $f(n) = \binom{n+2}{3}$ for non-negative integers $n$.

\begin{enumerate}[label=(\alph*)]
\item Show that $f(n) = \displaystyle\frac{(n+2)(n+1)n}{6}$. \hfill [2]
\item Find $f(0)$, $f(1)$, $f(2)$, $f(3)$, and $f(4)$. \hfill [2]
\item Show that $f(n) - f(n-1) = \displaystyle\frac{n(n+1)}{2}$ for $n \geq 1$. \hfill [4]
\item Hence show that $1 + 3 + 6 + 10 + \cdots + \displaystyle\frac{n(n+1)}{2} = \frac{n(n+1)(n+2)}{6}$. \hfill [3]
\end{enumerate}

\vspace{2em}

%% QUESTION 8: Arrangements + Coordinate Geometry
\textbf{Question 8} \hfill \textbf{[12 marks]}

Points are placed at every lattice point $(x, y)$ where $x$ and $y$ are integers with $0 \leq x \leq 4$ and $0 \leq y \leq 4$. This creates a $5 \times 5$ grid of 25 points.

\begin{enumerate}[label=(\alph*)]
\item Find the number of ways of choosing 3 points from the grid. \hfill [2]
\item Find the number of ways of choosing 3 collinear points that lie on a horizontal or vertical line. \hfill [3]
\item The number of ways of choosing 3 collinear points that lie on a diagonal line with gradient 1 or $-1$ is $k$. Show that $k = 16$. \hfill [4]
\item Hence find the number of triangles that can be formed using 3 of the 25 points as vertices. \hfill [3]
\end{enumerate}

\vspace{2em}

%% QUESTION 9: Binomial + Integration
\textbf{Question 9} \hfill \textbf{[12 marks]}

\begin{enumerate}[label=(\alph*)]
\item Expand $(1 - x^2)^4$ in ascending powers of $x$. \hfill [3]
\item Hence find $\displaystyle\int_0^1 (1-x^2)^4 \, dx$. \hfill [4]
\item The integral $\displaystyle\int_0^1 (1-x^2)^n \, dx = I_n$ for positive integers $n$.\\
Given that $I_1 = \frac{2}{3}$ and $I_2 = \frac{8}{15}$, show that $I_n = \displaystyle\frac{2^{2n}(n!)^2}{(2n+1)!}$. \hfill [5]
\end{enumerate}

\vspace{2em}

%% QUESTION 10: Permutations + Series
\textbf{Question 10} \hfill \textbf{[11 marks]}

\begin{enumerate}[label=(\alph*)]
\item Show that $\displaystyle\frac{1}{r!} - \frac{1}{(r+1)!} = \frac{r}{(r+1)!}$. \hfill [2]
\item Hence find $\displaystyle\sum_{r=1}^{n} \frac{r}{(r+1)!}$, giving your answer in terms of $n$. \hfill [4]
\item Evaluate $\displaystyle\sum_{r=1}^{\infty} \frac{r}{(r+1)!}$. \hfill [2]
\item Find $\displaystyle\sum_{r=1}^{\infty} \frac{r^2}{(r+1)!}$. \hfill [3]
\end{enumerate}

\vspace{2em}
\hrule
\vspace{0.5em}
\begin{center}
\textit{End of Questions}
\end{center}

\end{document}
